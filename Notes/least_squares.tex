\documentclass[12pt,a4paper]{article} % Default style and margins

% \usepackage[a4,center,noinfo,cross, width=16.5cm,height=24.5cm]{crop}


\usepackage{amsmath}
\usepackage{amssymb}
\usepackage{amsthm}
\usepackage{bm}
\usepackage{listings}

\newtheorem*{lemma}{Lemma}
\newtheorem*{theorem}{Theorem}

\DeclareMathOperator{\Rank}{Rank}
\DeclareMathOperator{\Colsp}{colsp}
\DeclareMathOperator{\Rowsp}{rowsp}
\DeclareMathOperator{\Range}{Range}
\DeclareMathOperator{\Span}{Span}
\DeclareMathOperator{\Null}{Null}

\title{Least Squares and the Pseudoinverse}
\author{Kenton Lam}

\begin{document}
\maketitle

The goal of this is to explain how the (Moore-Penrose) pseudoinverse relates 
to the last squares problem. Specifically, the psuedoinverse allows one to 
compute the least-norm solution to the problem.

\subsection*{Notation}
$\mathbf A \in \mathbb R^{m \times n}$ denotes an $m \times n$ matrix 
and $\mathbf A^\dagger$ is its pseudoinverse. $\Range \mathbf A$ is the vector space 
spanned by the columns of $\mathbf A$ (the column space). As a consequence, $\Range \mathbf A^\top$ is 
the span of the rows of $\mathbf A$ (the row space). $\Null \mathbf A$ is the null space of $\mathbf A$.

\subsection*{Fundamental Theorem of Linear Algebra}
\begin{theorem}
    If $\mathbf A \in \mathbb C^{m \times n}$, then 
    \begin{align*}
        \Null (\mathbf A) = \Range (\mathbf A^*)^\perp \text{ and }
        \Null (\mathbf A^*) = \Range(\mathbf A)^\perp
    \end{align*}
    where $\mathbf B^\perp$ denotes the orthogonal complement of $\mathbf B$ in its 
    containing vector space.
\end{theorem}

\section{Least Squares}
The ordinary least squares problems is the following.

Given a matrix $\mathbf A \in \mathbb R^{m \times n}$ 
and a vector $\bm b \in \mathbb R^{n}$, solve
\begin{align*}
    \min_{\bm x \in \mathbb R^n} \frac{1}{2} \| \mathbf A \bm x - \bm b \| ^2.
\end{align*}
The vector $\bm x$ is called the solution to this least squares problem. Note that this 
solution may not be unique.

\subsection{Pseudoinverse}
First, let's discuss the pseudoinverse.
It can be shown that there exists a unique $\mathbf A^\dagger \in \mathbb R^{n \times m}$ satisfying the Moore-Penrose conditions:
\begin{enumerate}
    \item $\mathbf A \mathbf A^\dagger \mathbf A = \mathbf A$
    \item $\mathbf A^\dagger \mathbf A \mathbf A^\dagger = \mathbf A^\dagger$
    \item $(\mathbf A \mathbf A^\dagger)^\top = \mathbf A \mathbf A^\dagger$ (so $\mathbf A \mathbf A^\dagger$ is symmetric)
    \item $(\mathbf A^\dagger \mathbf A)^\top = \mathbf A^\dagger \mathbf A$ (so $\mathbf A^\dagger \mathbf A$ is symmetric)
\end{enumerate}
Firstly, we show that $\mathbf A \mathbf A^\dagger$ is a projection matrix:
\begin{align*}
    (\mathbf A \mathbf A^\dagger) (\mathbf A \mathbf A^\dagger) = (\mathbf A \mathbf A^\dagger \mathbf A) \mathbf A^\dagger = \mathbf A \mathbf A^\dagger.
\end{align*}
Because it is symmetric, it must also be an orthogonal projection onto its range.

Now, we claim that $\Range \mathbf A^\dagger = \Range \mathbf A^\top$. Because $\Range \mathbf A^\top \perp \Null \mathbf A$, this is 
equivalent to saying $\Range \mathbf A^\dagger \perp \Null \mathbf A$. Suppose $\mathbf A^\dagger \bm x \in \Range \mathbf A^\dagger$ and $\bm y \in \Null \mathbf A$. Then,
to show they are orthogonal,
\begin{align*}
    (\mathbf A^\dagger \bm x)^\top \bm y &= \bm x^\top (\mathbf A^\dagger)^\top \bm y \\ 
    &= \bm x^\top (\mathbf A^\dagger \mathbf A\mathbf A^\dagger)^\top \bm y& \text{(using property of pseudoinverse)}&\\ 
    &= \bm x^\top ((\mathbf A^\dagger \mathbf A)\mathbf A^\dagger)^\top \bm y \\ 
    &= \bm x^\top (\mathbf A^\dagger)^\top (\mathbf A^\dagger\mathbf A)^\top \bm y \\ 
    &= \bm x^\top  (\mathbf A^\dagger)^\top \mathbf A^\dagger\mathbf A \bm y & \text{(because $\mathbf A^\dagger \mathbf A$ is symmetric)}\\ 
    &= \bm 0 & \text{(because $\bm y \in \Null \mathbf A$)}
\end{align*}
An identical process shows that $\Null \mathbf A^\dagger \perp \Range \mathbf A$ and hence
$\Null \mathbf A^\dagger = \Null \mathbf A^\top$.

\newpage
\subsection{Minimising the norm}
The goal is to solve for $\bm x$ which minimises
\begin{align*}
    \frac{1}{2}\| \mathbf A \bm x - \bm b \| ^2.
\end{align*}
Looking only at the norm, we can decompose $\bm b$ into two orthogonal vectors---one in $\Null \mathbf A^\top$ and one 
in $\Range \mathbf A$. This is done using the orthogonal projection $\mathbf A\mathbf A^\dagger$ we found earlier. 
It should be clear that this has not changed the equation. 
\begin{align*}
    \| \mathbf A \bm x - \bm b \| ^2 &= \| \mathbf A \bm x - (\mathbf A \mathbf A^\dagger + (\mathbf I - \mathbf A\mathbf A^\dagger))\bm b \|^2
\end{align*}
Now, we expand and regroup terms to get 
\begin{align*}
    \| \mathbf A \bm x - \bm b \| ^2 &= \| \underbrace{(\mathbf A \bm x - \mathbf A \mathbf A^\dagger \bm b)}_{\in \,\Range \mathbf A} + \underbrace{(\mathbf I - \mathbf A\mathbf A^\dagger)\bm b}_{\in \, \Null \mathbf A^\top} \|^2.
\end{align*}
It is obvious why the left part is in $\Range \mathbf A$. Recall that $\mathbf A \mathbf A^\dagger$ is an orthogonal projection onto $\Range \mathbf A$.
$\mathbf I - \mathbf A \mathbf A^\dagger$ projects onto the orthogonal complement of this, which is the $\Null \mathbf A^\top$.

Because $\Null \mathbf A^\top \perp \Range \mathbf A$, we can use Pythagoras' theorem to split the norm.
\begin{align*}
    \| {(\mathbf A \bm x - \mathbf A \mathbf A^\dagger \bm b)} + {(\mathbf I - \mathbf A\mathbf A^\dagger)\bm b} \|^2
    &= \| {\mathbf A \bm x - \mathbf A \mathbf A^\dagger \bm b} \| ^2 + \| {(\mathbf I - \mathbf A\mathbf A^\dagger)\bm b}  \| ^2
\end{align*}
The rightmost component is a constant so we can't change it. In fact, if $\mathbf A \bm x = \bm b$ has no exact solution, 
it is precisely because this part is non-zero. For our purposes of least squares, we can ignore it because 
\begin{align*}
    \arg\min \| \mathbf A \bm x - \bm b \| ^2 &= \arg \min \| {\mathbf A \bm x - \mathbf A \mathbf A^\dagger \bm b} \| ^2
\end{align*}
The value of the minimum is 0. It's important to note that this can be zero when 
the original least squares might not be 0. This is beacuse $\mathbf A \mathbf A^\dagger$ is an orthogonal projection onto 
$\Range \mathbf A$, so there always exists  an $\bm x$ such that $\mathbf A \bm x = \mathbf A \mathbf A^\dagger \bm b$ and the 
norm is 0, namely $\bm x = \mathbf A^\dagger \bm b$.

Contrast this with the original $\| \mathbf A \bm x - \bm b \|$. If $\bm b \notin \Range \mathbf A$, this will never be 0.
However, we can minimise it, which is what we aim to do with least squares. 

Finally, we want show that $\bm x = \mathbf A^\dagger \bm b$ 
is the solution with the smallest norm. Because $\bm x \in \Range \mathbf A^\top$, 
$\bm x$ has no components in the direction of  the null space. This is shown by 
\begin{align*}
    \bm x=\mathbf A^\dagger \bm b &= (\mathbf A^\dagger \mathbf A)\mathbf A^\dagger\bm b + \underbrace{(\mathbf I - \mathbf A^\dagger \mathbf A)\mathbf A^\dagger}_{=\,\bm 0}\bm b.
\end{align*}
If we add some component from the null space, we will get other solutions (because $\mathbf A \bm x$ will not change). 
These extra solutions can be expressed as
\begin{align*}
    \bm x &= \mathbf A^\dagger\bm b + \underbrace{(\mathbf I - \mathbf A^\dagger \mathbf A)\bm y}_{\in \,\Null \mathbf A}, \quad \text{ for any }y \in \mathbb R^m.
\end{align*}
Clearly, these solutions will have a larger norm than the original $\mathbf A^\dagger \bm b$, so we conclude 
$\mathbf A^\dagger \bm b$  is the solution with the smallest norm. \\[1em]
Corollary: If $\Null \mathbf A = \left\{ \bm 0\right\}$, then $\mathbf A^\dagger \bm b = \mathbf A^{-1} \bm b$ is the unique solution.

\end{document}