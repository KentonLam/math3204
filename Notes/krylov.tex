\documentclass[12pt,a4paper]{article} % Default style and margins

% \usepackage[a4,center,noinfo,cross, width=16.5cm,height=24.5cm]{crop}


\usepackage{amsmath}
\usepackage{amssymb}
\usepackage{amsthm}
\usepackage{bm}
\usepackage{listings}

\newtheorem*{lemma}{Lemma}
\newtheorem*{theorem}{Theorem}
\newtheorem*{definition}{Definition}

\DeclareMathOperator{\Rank}{Rank}
\DeclareMathOperator{\Colsp}{colsp}
\DeclareMathOperator{\Rowsp}{rowsp}
\DeclareMathOperator{\Range}{Range}
\DeclareMathOperator{\Span}{Span}
\DeclareMathOperator{\Null}{Null}

\title{Krylov Subspace Methods}
\author{Kenton Lam}

\begin{document}
\maketitle

This document aims to describe Krylov subspace methods, 
one of the best options for solving linear systems of equations.
Briefly, they have the following desirable properties:
\begin{itemize}
    \item No explicit form of $\mathbf A$ is needed; only a matrix-vector product is required.
    \item Well-suited for large and sparse systems.
    \item Optimised variations of Krylov methods are available for specific matrix types.
    \item For approximate solutions, Krylov methods have good convergence/approximation properties.
\end{itemize}
This is based on MATH3204 lectures and notes, lectured by Fred Roosta. 
The lecure slides contain proofs of some theorems not proved here.

\section{Introduction}
The general form of a linear system is 
\begin{align*}
    \mathbf A \bm x^\star &= \bm b
\end{align*}
where $\mathbf A \in \mathbb C^{ \times n}$ and $\mathbf A$ is invertible. 
Assume $\rho(\mathbf I - \mathbf A) < 1$. Then, we can write
$\mathbf A^{-1}$ as a geometric series,
\begin{align*}
    \mathbf A^{-1} = (\mathbf I - (\mathbf I - \mathbf A))^{-1} = \sum_{k=0}^\infty (\mathbf I-\mathbf A)^k.
\end{align*}
Suppose we have an initial guess $\bm x_0 \in \mathbb C^n$.
Define the residual of this guess as $\bm r_0 = \mathbf A \bm x^\star - \mathbf A \bm x_0$.
Then,
\begin{align*}
    \bm x^\star &= \mathbf A^{-1}\bm b = \mathbf A^{-1} 
    (\mathbf A \bm x^\star - \mathbf A \bm x_0 + \mathbf A \bm x_0) \\ 
    &= \bm x_0 + \mathbf A^{-1} \bm r_0 = \bm x_0 + \sum_{k=0}^\infty (\mathbf I-\mathbf A)^k \bm r_0
\end{align*}
This is great, but largely useless if we need to compute infinitely many vectors 
to find $\bm x^\star$. However it turns out that we actually don't need to.
\begin{theorem}[Cayley--Hamilton]
    Let $p_n(\lambda) = \sum_{i=0}^n c_i \lambda^i$ be the characteristic 
    polynomial of the matrix $\mathbf A$. Then, $p_n(\mathbf A) = 0$.
\end{theorem}
This implies that $\mathbf A^{-1}$ can be written as a finite sum of linear 
combinations of powers of $\mathbf A$. Specifically, it is a matrix polynomial 
of degree at most $n-1$. As a result, 
\begin{align*}
    \bm x^\star \in \bm x_0 + \Span \left\{ \bm r_0, \mathbf A \bm r_0, \ldots, \mathbf A^{n-1}\rm r_0\right\}.
\end{align*}
Suppose we only consider a subspace of this, so choose $k < n$
\begin{align*}
    \bm x_k \in \bm x_0 + \Span \left\{ \bm r_0, \mathbf A \bm r_0, \ldots, \mathbf A^{k-1}\rm r_0\right\}.
\end{align*}
This is the central question of Kylov methods. How good is the approximation $\bm x_k$ to 
$\bm x^\star$? What does ``good'' even mean?

In fact, Richardson iterations is a case of these subspace approximations. In Richardson, 
\begin{align*}
    \bm x_k = \bm x_0 + \sum_{i=0}^{k-1} \alpha_i \prod_{j=0}^{i-1} (\mathbf I - \alpha_j \mathbf A) \bm r_0
\end{align*}
However, depending on our choice of $\alpha_k$, we saw dramatically different convergence.

The great quest of Krylov subspace methods is to find the the ``best'' 
(in some sense)
$\bm x_k \approx \bm x^\star$ for some $k \ll n$.

\begin{definition}[Krylov Subspace]
    The Krylov subspace of order $k$, generated by the matrix $\mathbf A$ and 
    vector $\bm v$ is defined as 
    \begin{align*}
        \mathcal K_k(\mathbf A, \bm v) = \Span \left\{ \bm v, \mathbf A \bm v, 
        \ldots, \mathbf A^{k-1}\bm v\right\}
    \end{align*}
    for $k \ge 1$ and $\mathcal K_0(\mathbf A, \bm v) = \left\{ \bm 0\right\}$.
\end{definition}
Because these subspaces are nested, their dimensions cannot grow indefinitely.
At some point, the Krylov subspace will be large enough that it 
``contains'' all the information we can extract from $\mathbf A$ through its 
multiplication by $\bm v$. Consider the simplest case when $\bm v$ is an eigenvector, 
then the Kyrlov space just has dimension 1 for all $k$.
\begin{theorem}[Grade of $\bm v$ with respect to $\mathbf A$]
    There exists a positive integer $t = t(\bm v, \mathbf A)$, the grade of 
    $\bm v$ with respect to $\mathbf A$ such that 
    \begin{align*}
        \dim \mathcal K_k (\mathbf A, \bm v) = \min \left\{ k, t\right\}.
    \end{align*}
\end{theorem}
This means that for any $k \le t$, all the generated vectors are linearly independent. After 
$t$, the new vectors are linearly dependent on the previous ones. This means that 
for $k > t$, $\mathcal K_k (\mathbf A, \bm v) = \mathcal K_{k+1} (\mathbf A, \bm v)$.
As a direct corollary of this, 
\begin{align*}
    t = \min \left\{ k ~|~ \mathbf A^{-1}\bm v \in \mathcal K_k (\mathbf A, \bm v)\right\}.
\end{align*}
Recall that initially we had 
$\bm x^\star \in \bm x_0 + \mathcal K_n (\mathbf A, \bm r_0)$.
Now, we have a more specific result that 
\begin{align*}
    \bm x^\star \in \bm x_0 + \mathcal K_t (\mathbf A, \bm r_0)
\end{align*}
where $\bm r_0 = \bm b - \mathbf A - \bm x_0$ and $t$ is the grade of 
$\bm r_0$ with respect to $\mathbf A$.

To summarise, standard Krylov subspace solvers can be descibed as follows.
\begin{definition}[Standard Krylov Subspace Method]
    A standard Krylov subspace method is an {iterative method}, 
    which starting from some $\bm x_0$, 
    generates an appropriate sequence of iterates 
    $\bm x_k \in \bm x_0 + \mathcal K_k (\mathbf A, \bm r_0)$
    until it finds $\bm x^\star$ in exactly $t$ steps.

    The iterates are chosen appropriately such that if we terminate
    early, we have still $\bm x_k \approx \bm x^\star$ in some sense.
\end{definition}
Note that not all Krylov methods are of this form. Some are bulid upon 
different types of subspace or work with multiple subspaces (e.g.\ they 
also consider $\mathcal K_k (\mathbf A^*, \bm w)$).

Many terms are intentionally left vague in the above definition, because 
Krylov subspace solders differ among themselves in many aspects, such as 
\begin{itemize}
    \item the underlying Krylov subspace, 
    \item the method in which $\bm x_k$ is chosen, and
    \item the sense in which $\bm x_k \approx \bm x^\star$ is measured.
\end{itemize}
Additionally, in exact arithmetic, Krylov methods have finite termination property 
(i.e.\ they will always finish with an exact solution in finite iterations). 
Unfortunately, this does not hold in finite-precision
 arithmetic, such as on a computer.

\section{Computing a Basis}
\subsection{Motivation}
How do we construct vectors from some vector space? With a basis for that 
space, of course!

Suppose the grade of $\bm r_0$ w.r.t.\ $\mathbf A$ is $n$, so the basis matrix 
\begin{align*}
    \mathbf K = \begin{bmatrix}
        \bm r_0 & \mathbf A \bm r_0 & \cdots & \mathbf A^{n-1}\bm r_0
    \end{bmatrix} \in \mathbb R^{n \times n}
\end{align*}
is invertible. Then, 
\begin{align*}
    \mathbf A \mathbf K &= \begin{bmatrix}
        \mathbf A\bm r_0 & \mathbf A^2 \bm r_0 & \cdots & \mathbf A^{n}\bm r_0
    \end{bmatrix} \\ 
    &= \mathbf K \underbrace{\begin{bmatrix}
        \bm e_2 & \bm e_3 & \cdots & \bm e_n & \mathbf K^{-1} \mathbf A^n \bm r_0
    \end{bmatrix}}_{\mathbf C \in \mathbb R^{n \times n}}
\end{align*}
By construction, $\mathbf K^{-1} \mathbf A \mathbf K = \mathbf C$. It can be seen that 
$\mathbf C$ is an $n \times n$ matrix and upper Hessenberg.
Although $\mathbf C$ is sparse and easy to work with, such a basis is practically
useless for our purposes. 
\begin{itemize}
    \item Because $\mathbf C$ is $n \times n$, we need $n$ matrix-vector products.
    \item $\mathbf K$ could be very dense even if $\mathbf A$ is sparse. 
    \item $\mathbf K$ is ill-conditioned.
\end{itemize}

Suppose we take the $\mathbf Q \mathbf R$ decomposition of $\mathbf K$, so 
$\mathbf {K} = \mathbf {QR}$ where $\mathbf Q$ is orthogonal and $\mathbf R$ is 
upper triangular. Then, 
\begin{align*}
    \mathbf{Q}^{\top} \mathbf{A} \mathbf{Q}=\mathbf{R} \mathbf{K}^{-1} \mathbf{A} \mathbf{K} \mathbf{R}^{-1}=\mathbf{R} \mathbf{C} \mathbf{R}^{-1} = \mathbf{H}
\end{align*}
where $\mathbf H$ is an upper Hessenberg matrix. 
It can be seen that $\Range \mathbf K$ is the same as $\Range \mathbf Q$,
 so $\mathbf Q$ also 
spans our Krylov subspace.

For the subspace $\mathcal K_k (\mathbf A, \bm r_0)$, $k \ll n$, 
we search for $\mathbf Q_k \in \mathbb R^{n \times k}$ such that 
\begin{align*}
    \mathbf Q_k^\top \mathbf A \mathbf Q_k = \mathbf H_k \in \mathbb R^{k \times k}
\end{align*}
is upper Hessenberg. Note that this $\mathbf H_k$ is only $k \times k$ so 
all our computations can be done with this smaller matrix. To summarise, 
we aim to find $\mathbf Q_k$ with the following properties:
\begin{itemize}
    \item The columns  of $\mathbf Q_k$ form an orthonormal basis of $\mathcal K_k (\mathbf A, \bm r_0)$.
    \item $\mathbf Q_k^\top \mathbf A \mathbf Q_k = \mathbf H_k$ is upper Hessenberg.
\end{itemize}
In general, $\mathbf A \mathbf Q_k \ne \mathbf Q_k \mathbf H_k$ for any $k < n$.
Why is this so? Suppose we left-multiply $\mathbf Q_k^\top \mathbf A \mathbf Q_k = \mathbf H_k$
 by $\mathbf Q_k$. 
If this was orthgonal, we'd have $\mathbf Q_k \mathbf Q_k^\top = \mathbf I_n$.
However, the matrix product  $\mathbf Q_k \mathbf Q_k^\top$ is essentially 
a map $\mathbb R^n \to \mathbb R^k \to \mathbb R^n$. If this is identity, 
it implies a $k$-dimensional set can span an $n$-dimensional space, 
which is absurd since $k < n$.

To get equality, we adjust with an error term $\mathbf E_k \in \mathbb R^{n \times k}$,
\begin{align*}
    \mathbf A \mathbf Q_k = \mathbf Q_k \mathbf H_k + \mathbf E_k.
\end{align*}
In order to have $\mathbf Q_k^\top \mathbf A \mathbf Q_k = \mathbf H_k$ 
hold, we need $\mathbf Q_k^\top \mathbf E_k = \mathbf 0$.

Suppose we have a vector $\bm q_{k+1}$, orthogonal to all $q_i \le k$.
Then, if $\mathbf E_k = \bm q_{k+1} \bm h^\top_{k}$ for any $\bm h_k \in \mathbb R^n$. 
Because $\bm q_{k+1}$ is orthogonal to every column of $\mathbf Q_k^\top$, we 
see that $\mathbf Q_k^\top \mathbf E_k = \mathbf Q_k^\top \bm q_{k+1} \bm h^\top_{k} = \mathbf 0$. 
Because this holds for any $\bm h_k$, we choose $\bm h_k$ with zeros in all positions 
except the $k$-th, where it is $h_{k+1,k}$.

So $\mathbf{A} \mathbf{Q}_{k}=\mathbf{Q}_{k} \mathbf{H}_{k}+\bm{q}_{k+1} \bm{h}_{k}^{\top},$ and we can write
\begin{align*}
    \mathbf{A} \mathbf{Q}_{k}=
    \underbrace{\begin{bmatrix}
        \mathbf Q_k & \bm q_{k+1}
    \end{bmatrix}}_{\mathbf Q_{k+1}}
    \underbrace{\begin{bmatrix}\mathbf{H}_{k} \\[0.8em] \bm{h}_{k}^{\top}\end{bmatrix}}_{\mathbf H_{k+1,k}}, 
        \quad \text { where }\bm{h}_{k}^\top= \begin{bmatrix}
            0 & \cdots & 0 & h_{k+1,k}
        \end{bmatrix}.
\end{align*}
This gives us an expression for $\bm q_{k+1}$ given all the previous $\bm q_k$'s.

\subsection{Arnoldi Process}
The Arnoldi algorithm is a modified version of Gram-Schmidt which finds 
the desired $\mathbf Q_k$.

In the base case of $k=1$, 
we just have $\bm q_1 = \bm r_0 / \| \bm r_0 \|$.

For $k=2$, we have 
\begin{align*}
    \mathbf{A} \bm q_1=
    \begin{bmatrix}
        \bm q_1 & \bm q_2
    \end{bmatrix}
    \begin{bmatrix}
        h_{11} \\ h_{21}
    \end{bmatrix} \implies 
    \mathbf A \bm q_1 = h_{11} \bm q_1 + h_{21} \bm q_2.
\end{align*}
We apply the fact that $\mathbf Q_k$ must have orthonormal columns. 
\begin{align*}
    \bm q_1^\top\mathbf A \bm q_1 = h_{11} \bm q_1^\top\bm q_1 + h_{21} \bm q_1^\top\bm q_2&& \implies h_{11} &= \langle \bm q_1, \mathbf A \bm q_1\rangle \\ 
    \mathbf A \bm q_1 - h_{11} \bm q_1 = h_{21} \bm q_2&& \implies h_{21} &= \| \mathbf A \bm q_1 - h_{11} \bm q_1\| \\ 
    &&\implies \,\,\bm q_2 &= \frac{\mathbf A \bm q_1 - h_{11} \bm q_1}{h_{21}}
\end{align*}

Consider the general case of $k=j$. Observe that the $j$-th step adds 
one row and one column to $\mathbf H_{j+1,j}$, namely the $(j+1)$-th row and $j$-th 
column. As block matrices, this can be visualised as 
\begin{align*}
    \mathbf A \mathbf Q_j &= \begin{bmatrix}
        \mathbf Q_j & \bm q_{j+1}
    \end{bmatrix}
    \begin{bmatrix}
        \mathbf H_{j,j-1} & \begin{matrix}
            h_{1j} \\ \vdots \\ h_{jj}
        \end{matrix} \\ 
        \begin{matrix}
            0 & \cdots & 0
        \end{matrix} & h_{j+1,j}
    \end{bmatrix}
\end{align*}
Because we only change the last columns and rows, this can be reduced to 
\begin{align*}
    \mathbf A \bm q_j = h_{1j}\bm q_1 + h_{2j} \bm q_2 + \cdots + h_{j+1,j}\bm q_{j+1}
\end{align*}



\end{document}